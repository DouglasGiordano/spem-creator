\begin{resumo}%[\textsc{Resumo}]
 
 Comumente antes do desenvolvimento de um software um processo de software deve ser definido para guiar todo o desenvolvimento do mesmo. Atualmente existem vários tipos de software que oferecem suporte as mais diferentes etapas de um processo de software. A partir de um mapeamento sistemático podemos visualizar em uma escala mais abrangente a necessidade de um software que apoie a execução do processo de software, em um nível que ofereça também um suporte a gestão de todos seus artefatos gerados em cada etapa de sua execução. O objetivo deste trabalho é oferecer uma ferramenta de apoio a execução de processos modelados em SPEM. Para isso construímos uma ferramenta que a partir de um documento no formato SPEM pode gerar um diagrama que possibilita o acompanhamento da execução de cada etapa do processo e também o arquivamento dos artefatos gerados em cada etapa.
 

\end{resumo}
