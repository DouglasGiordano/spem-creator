%==============================================================================
\section{Introdução}\label{introducao}
%==============================================================================
A demanda de software atualmente cresce cada vez mais. Em todo lugar podemos perceber softwares de todos os tipos em diferentes dispositivos. Isso gera uma grande demanda de produtos de software. Com a produção de software cada vez mais agressiva, a necessidade de técnicas que auxiliem a produção de software em menos tempo e com maior qualidade é requirida\cite{larman2007utilizando}.

Processo de software de software está bastante ligado com a qualidade e tempo de construção de um software, estando relacionado com o sucesso ou fracasso de projeto\cite{sommerville2007engenharia}. 

Comumente antes do desenvolvimento de um software, o processo de como esse produto será construído é definido. A definição é apenas uma das etapas relacionadas ao processo de software, sendo ligada a outras características como por exemplo a melhoria e a execução do mesmo.

Com objetivo de oferecer uma ferramenta de auxilio a etapa de execução de processos, esse artigo apresenta a construção de uma ferramenta. Essa ferramenta auxilia nas principais etapas da execução do processo, como o controle de fluxo das atividades, e controle dos artefatos de softwares gerados em cada uma delas.

Na seção \ref{fundamentacaoTeorica} é apresentado os principais conceitos envolvidos na modelagem de processos. Na seção \ref{sec:trabalhosRelacionados} é apresentado um mapeamento das ferramentas que auxiliam algumas etapas do processo de software. Já na seção \ref{ferramentaDeModelagemColabvorativa} é descrito alguma das etapas do desenvolvimento da ferramenta e as principais estruturas envolvidas na implementação. Por ultimo na seção \ref{resultados} é mostrado a ferramenta, e um pouco de suas funcionalidades.